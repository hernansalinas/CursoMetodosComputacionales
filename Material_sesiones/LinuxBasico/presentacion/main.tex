\documentclass{beamer}

\title{Introducción a los Comandos Básicos de Linux en Bash}
\author{Tu Nombre}
\date{\today}

\begin{document}

\frame{\titlepage}

\begin{frame}
\frametitle{Objetivos}
\begin{itemize}
    \item Aprender los comandos básicos para la navegación en Linux.
    \item Manipular archivos y directorios desde la consola.
    \item Obtener información y gestionar el sistema: memoria, disco duro, procesador, y puertos.
    \item Gestionar permisos de archivos y directorios.
\end{itemize}
\end{frame}

\begin{frame}
\frametitle{Navegación en el Sistema de Archivos}
\begin{itemize}
    \item \textbf{pwd}: Muestra la ruta del directorio actual.
    \begin{itemize}
        \item \textbf{Sintaxis}: `pwd`
        \item \textbf{Ejemplo}: Muestra `/home/usuario` si estás en ese directorio.
    \end{itemize}
    \item \textbf{ls}: Lista los archivos y directorios en el directorio actual.
    \begin{itemize}
        \item \textbf{Sintaxis}: `ls [opciones] [directorio]`
        \item \textbf{Ejemplo}: `ls -l` muestra una lista detallada de archivos.
    \end{itemize}
    \item \textbf{cd}: Cambia de directorio.
    \begin{itemize}
        \item \textbf{Sintaxis}: `cd [directorio]`
        \item \textbf{Ejemplo}: `cd /etc` te lleva al directorio `/etc`.
    \end{itemize}
    \item \textbf{tree}: Muestra la estructura de directorios y archivos de manera jerárquica.
    \begin{itemize}
        \item \textbf{Sintaxis}: `tree [directorio]`
        \item \textbf{Ejemplo}: `tree /home` muestra la estructura de directorios en `/home`.
    \end{itemize}
\end{itemize}
\end{frame}

\begin{frame}
\frametitle{Manipulación de Archivos y Directorios}
\begin{itemize}
    \item \textbf{cp}: Copia archivos o directorios.
    \begin{itemize}
        \item \textbf{Sintaxis}: `cp [opciones] origen destino`
        \item \textbf{Ejemplo}: `cp archivo.txt /backup/` copia `archivo.txt` a `/backup/`.
    \end{itemize}
    \item \textbf{mv}: Mueve o renombra archivos o directorios.
    \begin{itemize}
        \item \textbf{Sintaxis}: `mv [opciones] origen destino`
        \item \textbf{Ejemplo}: `mv archivo.txt documento.txt` renombra `archivo.txt` a `documento.txt`.
    \end{itemize}
    \item \textbf{rm}: Elimina archivos o directorios.
    \begin{itemize}
        \item \textbf{Sintaxis}: `rm [opciones] archivo`
        \item \textbf{Ejemplo}: `rm -r /directorio` elimina el directorio `/directorio` y su contenido.
    \end{itemize}
    \item \textbf{mkdir}: Crea un nuevo directorio.
    \begin{itemize}
        \item \textbf{Sintaxis}: `mkdir [opciones] nombre_directorio`
        \item \textbf{Ejemplo}: `mkdir proyectos` crea un directorio llamado `proyectos`.
    \end{itemize}
\end{itemize}
\end{frame}

\begin{frame}
\frametitle{Ver Información del Sistema}
\begin{itemize}
    \item \textbf{uname -a}: Muestra información del sistema operativo.
    \begin{itemize}
        \item \textbf{Sintaxis}: `uname -a`
        \item \textbf{Ejemplo}: Muestra información como el kernel, el nombre del host, etc.
    \end{itemize}
    \item \textbf{top}: Muestra los procesos en ejecución y su uso de recursos.
    \begin{itemize}
        \item \textbf{Sintaxis}: `top`
        \item \textbf{Ejemplo}: Muestra en tiempo real el uso de CPU, memoria y los procesos activos.
    \end{itemize}
    \item \textbf{htop}: Una versión mejorada de `top` con una interfaz más amigable.
    \begin{itemize}
        \item \textbf{Sintaxis}: `htop`
        \item \textbf{Ejemplo}: Proporciona una vista interactiva de los procesos y su uso de recursos.
    \end{itemize}
    \item \textbf{df -h}: Muestra el uso del espacio en disco de manera legible.
    \begin{itemize}
        \item \textbf{Sintaxis}: `df -h`
        \item \textbf{Ejemplo}: Muestra el uso del disco en formato legible (e.g., 10G).
    \end{itemize}
    
\end{itemize}
\end{frame}

\begin{frame}
\frametitle{Ver Información del Sistema}
\begin{itemize}
    \item \textbf{du -sh *}: Muestra el tamaño de archivos y directorios en el directorio actual.
    \begin{itemize}
        \item \textbf{Sintaxis}: `du -sh [archivo/directorio]`
        \item \textbf{Ejemplo}: `du -sh *` muestra el tamaño de cada archivo y directorio en el directorio actual.
    \end{itemize}
    
    \item \textbf{free -h}: Muestra la memoria libre y usada en el sistema.
    \begin{itemize}
        \item \textbf{Sintaxis}: `free -h`
        \item \textbf{Ejemplo}: Muestra la memoria usada y disponible en formato legible.
    \end{itemize}
    \item \textbf{vmstat}: Muestra estadísticas del sistema, incluyendo memoria, procesos, y CPU.
    \begin{itemize}
        \item \textbf{Sintaxis}: `vmstat [opciones]`
        \item \textbf{Ejemplo}: `vmstat 5` muestra estadísticas del sistema cada 5 segundos.
    \end{itemize}
\end{itemize}
\end{frame}



\begin{frame}
\frametitle{Búsqueda de Texto en Archivos}
\begin{itemize}
    \item \textbf{grep}: Busca patrones en archivos.
    \begin{itemize}
        \item \textbf{Sintaxis}: `grep [opciones] patrón [archivo]`
        \item \textbf{Ejemplo}: `grep "error" /var/log/syslog` busca la palabra "error" en el archivo `syslog`.
    \end{itemize}
    \item \textbf{find}: Busca archivos y directorios en un sistema de archivos.
    \begin{itemize}
        \item \textbf{Sintaxis}: `find [ruta] [opciones] [expresión]`
        \item \textbf{Ejemplo}: `find /home -name "*.txt"` busca todos los archivos `.txt` en el directorio `/home`.
    \end{itemize}
    \item \textbf{locate}: Encuentra archivos rápidamente utilizando una base de datos indexada.
    \begin{itemize}
        \item \textbf{Sintaxis}: `locate [nombre_archivo]`
        \item \textbf{Ejemplo}: `locate archivo.txt` busca `archivo.txt` en todo el sistema de archivos.
    \end{itemize}
\end{itemize}
\end{frame}

\begin{frame}
\frametitle{Procesamiento de Texto}
\begin{itemize}
    \item \textbf{awk}: Herramienta poderosa para la manipulación y procesamiento de texto.
    \begin{itemize}
        \item \textbf{Sintaxis}: `awk '{acción}' archivo`
        \item \textbf{Ejemplo}: `awk '{print $1, $3}' archivo.txt` imprime la primera y tercera columna del archivo `archivo.txt`.
    \end{itemize}
    \item \textbf{sed}: Editor de texto en línea para buscar y reemplazar.
    \begin{itemize}
        \item \textbf{Sintaxis}: `sed 's/patrón/reemplazo/' archivo`
        \item \textbf{Ejemplo}: `sed 's/error/warning/' archivo.txt` reemplaza "error" por "warning" en `archivo.txt`.
    \end{itemize}
    \item \textbf{cut}: Corta secciones de cada línea de archivos.
    \begin{itemize}
        \item \textbf{Sintaxis}: `cut [opciones] [archivo]`
        \item \textbf{Ejemplo}: `cut -d',' -f1,3 archivo.csv` extrae la primera y tercera columna de un archivo CSV separado por comas.
    \end{itemize}
\end{itemize}
\end{frame}

\begin{frame}
\frametitle{Combinación de Comandos}
\begin{itemize}
    \item \textbf{xargs}: Construye y ejecuta comandos desde la salida estándar.
    \begin{itemize}
        \item \textbf{Sintaxis}: `comando | xargs [otro_comando]`
        \item \textbf{Ejemplo}: `find . -name "*.log" | xargs rm` encuentra y elimina todos los archivos `.log`.
    \end{itemize}
    \item \textbf{sort}: Ordena las líneas de texto en archivos.
    \begin{itemize}
        \item \textbf{Sintaxis}: `sort [opciones] archivo`
        \item \textbf{Ejemplo}: `sort -n archivo.txt` ordena las líneas numéricamente en `archivo.txt`.
    \end{itemize}
    \item \textbf{uniq}: Reporta o elimina líneas duplicadas en un archivo ordenado.
    \begin{itemize}
        \item \textbf{Sintaxis}: `uniq [opciones] [archivo]`
        \item \textbf{Ejemplo}: `sort archivo.txt | uniq` elimina las líneas duplicadas de un archivo.
    \end{itemize}
    \item \textbf{wc}: Cuenta las líneas, palabras y caracteres en un archivo.
    \begin{itemize}
        \item \textbf{Sintaxis}: `wc [opciones] [archivo]`
        \item \textbf{Ejemplo}: `wc -l archivo.txt` cuenta las líneas en `archivo.txt`.
    \end{itemize}
\end{itemize}
\end{frame}


\begin{frame}
\frametitle{Gestión de Permisos, comandos más avanzados}
\begin{itemize}
    \item \textbf{chmod}: Cambia los permisos de archivos o directorios.
    \begin{itemize}
        \item \textbf{Sintaxis}: `chmod [opciones] permisos archivo`
        \item \textbf{Ejemplo}: `chmod 755 script.sh` otorga permisos de lectura, escritura y ejecución al propietario, y permisos de lectura y ejecución al grupo y otros usuarios.
    \end{itemize}
    \item \textbf{chown}: Cambia el propietario de archivos o directorios.
    \begin{itemize}
        \item \textbf{Sintaxis}: `chown [opciones] usuario:grupo archivo`
        \item \textbf{Ejemplo}: `chown root:root /etc/archivo.conf` cambia el propietario y el grupo del archivo a `root`.
    \end{itemize}
    \item \textbf{umask}: Establece la máscara de creación de archivos.
    \begin{itemize}
        \item \textbf{Sintaxis}: `umask [opciones]`
        \item \textbf{Ejemplo}: `umask 022` establece los permisos predeterminados para nuevos archivos como `755` y para directorios como `755`.
    \end{itemize}
\end{itemize}
\end{frame}


\begin{frame}
\frametitle{Gestión de Permisos}
\begin{itemize}


    
    \item \textbf{getfacl/setfacl}: Obtiene y establece listas de control de acceso (ACL) para archivos y directorios.
    \begin{itemize}
        \item \textbf{Sintaxis}: 
        \begin{itemize}
            \item `getfacl archivo`: Obtiene la ACL de un archivo.
            \item `setfacl -m usuario:permisos archivo`: Establece una ACL en un archivo.
        \end{itemize}
        \item \textbf{Ejemplo}: 
        \begin{itemize}
            \item `getfacl archivo.txt` muestra la ACL del archivo `archivo.txt`.
            \item `setfacl -m u:usuario:rwx archivo.txt` otorga permisos de lectura, escritura y ejecución al usuario `usuario` en `archivo.txt`.
        \end{itemize}
    \end{itemize}
\end{itemize}
\end{frame}

\begin{frame}
\frametitle{Conclusión}
\begin{itemize}
    \item Los comandos de Linux son poderosos y permiten un control completo sobre el sistema.
    \item Practicar estos comandos es esencial para cualquier usuario o administrador de sistemas.
    \item Explorar más comandos y combinaciones es la clave para mejorar tus habilidades en la consola.
\end{itemize}
\end{frame}

\end{document}
